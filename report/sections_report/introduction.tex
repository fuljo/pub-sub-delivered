\section{Introduction}
\label{sec:intro}

We consider a food delivery application with three types of users:
\begin{itemize}[noitemsep]
    \item \textbf{Customers}: can place orders and check their status 
    \item \textbf{Administrators}: can insert new products and change their availability
    \item \textbf{Delivery men}: can notify the successful delivery of an order that was in the shipping phase 
\end{itemize}
Each user has an identifier (string), a full name, an email, a shipping address and a \emph{role} (one of the categories above).

A \emph{product} has an identifier (string), a name, a description, a unitary price and might be available or not.
An \emph{order} is always associated to a \emph{customer}, contains a series of \emph{products} and a quantity for each of them.
It also has a \emph{state} which can be:
\begin{itemize}[noitemsep]
    \item \textbf{Created}: the order has just been created by the user, not yet processed 
    \item \textbf{Validated}: the order contains at least one product, and each product is available 
    \item \textbf{Shipping}: the order has been validated, and its shipping address is also valid, so it is being shipped to the customer
    \item \textbf{Shipped}: the order has been successfully shipped to the customer
    \item \textbf{Failed}: the order did not pass one of the validation steps
\end{itemize}

The application is divided into three microservices, that provide a REST interface to the frontend, which might be a native application or a web app.
\begin{itemize}
    \item The \textbf{Users} service provides facilities to register/view the users (of any kind) 
    \item The \textbf{Orders} service enables \emph{customers} to place orders and view their status, \emph{administrators} to insert products and change their availability.
    In the background, it also validates created orders and monitors the status of shipments to update the status of the orders.
    \item The \textbf{Shipping} service enables \emph{delivery men} to notify the delivery of orders, and it checks the address of orders before making them available for shipping.
\end{itemize}

We use Apache Kafka as the underlying framework to handle inter-service communication, as well as providing persistency and fault-tolerance to the application data.
In the following sections we provide a more in-depth look at how the application is designed internally, as well as the interface it exposes.